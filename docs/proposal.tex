\documentclass{article}

\usepackage{appendix}
\usepackage{fullpage}
\usepackage{syntax}
\usepackage{listings}
\usepackage{listings-golang}
\usepackage{color}

\definecolor{mygreen}{rgb}{0,0.6,0}
\definecolor{mygray}{rgb}{0.5,0.5,0.5}
\definecolor{mymauve}{rgb}{0.58,0,0.82}

\lstset{ % add your own preferences
  backgroundcolor=\color{white},   % choose the background color; you must add \usepackage{color} or \usepackage{xcolor}
  basicstyle=\footnotesize,        % the size of the fonts that are used for the code
  breakatwhitespace=false,         % sets if automatic breaks should only happen at whitespace
  breaklines=true,                 % sets automatic line breaking
  captionpos=t,                    % sets the caption-position to bottom
  commentstyle=\color{mygreen},    % comment style
  deletekeywords={...},            % if you want to delete keywords from the given language
  escapeinside={\%*}{*)},          % if you want to add LaTeX within your code
  extendedchars=true,              % lets you use non-ASCII characters; for 8-bits encodings only, does not work with UTF-8
  frame=top,frame=bottom,	                   % adds a frame around the code
  keepspaces=true,                 % keeps spaces in text, useful for keeping indentation of code (possibly needs columns=flexible)
  keywordstyle=\color{blue},       % keyword style
  otherkeywords={*,...},           % if you want to add more keywords to the set
  numbers=left,                    % where to put the line-numbers; possible values are (none, left, right)
  numbersep=5pt,                   % how far the line-numbers are from the code
  numberstyle=\tiny\color{mygray}, % the style that is used for the line-numbers
  rulecolor=\color{black},         % if not set, the frame-color may be changed on line-breaks within not-black text (e.g. comments (green here))
  showspaces=false,                % show spaces everywhere adding particular underscores; it overrides 'showstringspaces'
  showstringspaces=false,          % underline spaces within strings only
  showtabs=false,                  % show tabs within strings adding particular underscores
  stepnumber=2,                    % the step between two line-numbers. If it's 1, each line will be numbered
  stringstyle=\color{mymauve},     % string literal style
  tabsize=2,	                   % sets default tabsize to 2 spaces
  title=\lstname,                   % show the filename of files included with \lstinputlisting; also try caption instead of title
  literate={+}{{\texttt{+}}}1 {-}{-}1,
  language=Golang % this is it !
}

\title{Slower, weaker, buggier: Go-{}-}

\date{}
\author{Jeffrey A. Robinson}

\begin{document}

\maketitle

\section{Introduction}

Go is a strongly typed programming language focused towards systems programming and concurrency.
We hope to support a subset of features from the language such as records, loops, conditionals, and functions.
To ensure compliance with our expectations we include a list of programs we hope to support along with the grammar we plan on implementing.  
Our grammar will be based on the Go grammar provided by the Go team located at {\bf golang.org}.   


\section{Support}

In this section we give three levels of expectation on the features we want to support: expect support, might support, and unlikely support.  For those features we 
``expect to support'' we have high confidence we can incorporate them into our language,
for those labeled ``might support'' we feel that time constraints and complexity hinder completion,
and for ``unlikely to support'' we feel that the complexity and time constraints prohibit inclusion.

\subsection{Expect to Support}

We want our language and compiler to be able to support programs that might be given in an introductory programming class.
For this we have created four programs, which can be found in the appendix, that a freshman could write after taking a programming class.  
Of the four programs, we expect to be able to compile the first three programs, and if possible, the fourth.

\begin{itemize}
    \item (High Chance) Basic I/O
    \item (High Chance) Simple Calculator
    \item (High Chance) Iterative and Recursive Fibonacci
    \item (Low Chance)  Infix to Prefix conversion
\end{itemize}

\noindent In order for these example programs to compile we require the following programming constructs

\begin{itemize}
    \item primitives    - integers, reals, booleans, and strings
    \item conditionals  - if and else statements along with (possible) switch statements
    \item loops         - infinite, conditional, and range based
    \item functions     - non variadic functions with zero or more arguments and zero or one return values
    \item structures    - arrays 
\end{itemize} 

\subsection{Might Support}

If we finish all required items and have enough time to implement the following 

\begin{itemize}
    \item full support for structs  - Allow for grouping of data into a single structure
    \item variadic functions        - Allow for functions allow a variadic number of arguments.
    \item multiple return values    - Allow for zero, one, or more return values and types
\end{itemize}
 
We believe that most of these are not difficult to implement but that due to time constraints they might not be completed.

\subsection{Unlikely Support} 

In this section we talk about a list of items we want to support but will not have the time to support.

\begin{itemize}
    \item first class functions     - Allow for anonymous and named functions to created within a function.
    \item struct functions  - Functions for structures allow for implementation of interfaces.  However, we would need to support binding functions to structures and a method to call them.
    \item Interfaces        - Interfaces in Go use duck typing. This requires verification of structure function signatures  
    \item Imports           - Being able to handle file inclusion, dependency checks, and name collision. 
\end{itemize}

%\section{Semantics}

%Go supports static typing however it allows for untyped variables as in the snippet below
%
%\begin{lstlisting}[title=Untyped value]
%package main
%
%func main() {
%    i := 0
%    println(i)
%}
%\end{lstlisting}
%
%For this code snippet we have not defined the type of $i$ and we do not know if the value 0 is an $int$, $int32$, $int64$, or other numeric value.
%If the value of the right side can be determined then the type of $i$ can be infered.
%Otherwise, to ensure static typing we have the following rules
%
%\begin{itemize}
%   \item   If the expression on the right is an integral literal the type is $int$
%    \item   If the expression on the right is a floating point literal the type is $float$
%    \item   If the expression on the right is a string literal the type is $string$
%    \item   If the expression on the right is an array of values then the type is an array of the least restrictive type as defined above. 
%\end{itemize}

%For the semantics of our language we plan on following the Go language specification description.  
%For instance the ``short variable declaration'' includes no type and therefore must be inferred from the value assigned.


\part*{Appendix}

\begin{appendices}

\section{Required Programs}

\subsection{Basic I/O}

The Basic I/O function reads and writes three data types: floats, strings, and integers.
We read in two floating point numbers and output their sum, we read in the name of the user and output it, finally we read in an integer and output its double.

\begin{lstlisting}[title=Basic I/O Program]
package main

func main() {
    print("Please enter a number")
    x := readFloat()
    print("Please enter another number")
    y := readFloat()
    println("The sum of these two numbers is", x + y);

    print("Please enter your name")
    name := readString()
    println("Your name is", name)

    print("Please enter an integer")
    i := readInt()
    println("The integer doubled is",i + i)
}
\end{lstlisting}

\subsection{Basic Calculator}

Our basic calculator reads in a math expression from the console and evaluates it by reading left to right.
All operators have identical precendence and is left associative.  This shows that we support the operators: +, -, *, and /. 
This program runs until either an invalid operator is read or we try to read a real number and fail.
This function shows that we can handle multiple math operators, functions, and infinite loops. 


\begin{lstlisting}[title=Basic Calculator]
package main

func add(x , y float64) float64 {
    return x + y
}

func subtract(x , y float64) float64 {
    return x - y
}

func multiply(x , y float64) float64 {
    return x * y
}

func divide(x , y float64) float64 {
    return x / y
}

func main() {
    println("Welcome to the calculator")
    for {
        

        err := false
        errMsg := ""
        x := readFloat()
        op := readCharacter()
        y := readFloat()
        readCharacter() // newline

        switch op {
            case '+':
                res = add(res,y)
            case '-':
                res = subtract(res,y)
            case '*':
                res = multiply(res,y)
            case '/':
                if y != 0 {
                    res = divide(res,y)
                } else {
                    err = true
                    errMsg = "Divide by zero"
                }
            default:
                err = true
                errMsg = "Invalid operator"
        }
        if err {
            println(errMsg)
            return
        } else {
            println(res)
        }
    }
}
\end{lstlisting}

\subsection{Iterative and Recursive Fibonacci}

For this program we implement two functions that compute Fibonacci numbers.  These two 
programs show that we can handle iteration and recursion, and loops that handle a condition. 

\begin{lstlisting}[title=Iterative and Recursive Fibonacci]
package main

func recursiveFib(i int) int {
    if i < 2 {
        return 1
    }
    return fib(i-1)+fib(i-2)
}

func iterativeFib(i int) int {
    a := 1
    b := 1
    for i > 0 {
        t := a+b
        a = b
        b = t
        i--
    }
    return a
}

int main() {
    print(recursiveFib(10));
    print(iterativeFib(10));
}

\end{lstlisting}

\subsection{Infix to Prefix}

This program showcases multiple programming constructs which we believe will be difficult to implement.
This program takes in an infix mathematical operation and converts it into prefix notation.
We use four new constructs: structures, structure functions, arrays, and range loops.

\begin{lstlisting}[title=Infix to Prefix]
package main

type stack struct {
    stk [100]rune
    pos   int
}

func (s *stack) Push(v rune) {
    s.stk[s.pos++] = v
}

func (s *stack) Pop() rune {
    s.pos--
    return s.stk[s.pos]
}

func (s *stack) Empty() bool {
    return s.pos == 0
}

func (s *stack) Peek() rune {
    return s.stk[s.pos - 1]
}

// Ruturns true if a has higher precedence
func precedence(a , b rune) bool {
    if a == b {
        return true
    }
    if a == '+' || a == '-' {
        if b == '*' || b == '/' {
            return false
        }
        return true
    }
    return !precedence(b,a)
}

func isDigit(r rune) bool {
    return '0' <= r && r <= '9'
} 

func reverse(data []rune) {
    left := 0
    right := len(data) - 1
    for left < right {
        t := data[left]
        data[left] = data[right]
        data[right = t
        left++
        right--
    }
}

int main() {
    data := {'1', '+', '2', '*' , '5', '/', '2' , '+' , '4' , '-' , '3' }
    result := []rune(nil)
    
    // Step 1 - Reverse array
    reverse(data)

    // Step 2 - Convert to postfix
    stk := &stack{}
    for i := range data {
        if isOp(data[i]) {
            for !stk.Empty() {
                if precedence(stk.Peek(),data[i]) {
                    result = append(result,stk.Pop())
                }else {
                    break
                }
            }else {
            stk.Push(data[i])
        } else {
            result = append(result,data[i])
        }
    }

    for !stk.Empty() {
        result = append(result,stk.Pop())
    }

    // Step 3 - reverse answer
    reverse(result)

    // Display answer
    for i := range result {
        print(result[i])
    }
    println()
}

\end{lstlisting}


\end{appendices}

\end{document}
